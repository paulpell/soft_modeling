%
% Memory based reasoning - Project
% Specification
%
% created: 04. March 2012
%
\documentclass[11pt]{article}
\usepackage[T1]{fontenc}

% not needed here:
%\usepackage{graphicx}
%\usepackage{hyperref}
%\hypersetup{
%    colorlinks,
%    citecolor=black,
%    filecolor=black,
%    linkcolor=black,
%    urlcolor=black
%}

\usepackage{url}

\title{
	\emph{A Specification for}\\
	\huge{\textbf{Soft Object Modelling \& Rendering} }\\
	-Project-\\
	Computer Graphics\\[2em]	
}
\author{
	Philipp Fonteyn (MS11F010)\\
	Paul Pellegrini (CS11F001)\\[2em]
	\emph{Master of Technology}\\
	\emph{Computer Science \& Engineering, IIT Madras}
}
%\date{1st of March 2012}

%
% Document
%
\begin{document}
\maketitle
\newpage
%\tableofcontents
%\newpage

%
%
%
\section{Introduction}
In modern computer science the field of computer graphics has gained much impulse due to faster performing devices, better and quicker memory access times and of course more efficient methods of algorithmic computation. Thus more and more realistic and complex scenes can be rendered by computational means and finally displayed on digital devices. This project will give a small overview over the technique of "soft modeling".
%
%
%
\section{Modeling Soft Objects}
When modeling the world solid objects such as e.g. a cube have clearly defined static shapes and appearance in the three-dimensional space. So called "soft objects" on the other hand are more flexible and can take various shapes and forms. Storing softer objects and their interaction with their surroundings needs therefore more thought as the solids.\\[1em]
%
There have been several models developed for modeling soft objects. The following paper will focus on the spring model approach described by \cite{LSCS} and \cite{hair}\cite{gama}.

%
%
%
\section{Project Outlook}
%The project will cover several aspects such as:\\[1em]
%
\textbf{Description of wire-frame and spring model domain} - As the basic concept for three-dimensional display the wire-frame model will be explained and why we still need it for soft objects. The underlying theory behind the whole spring model will be explained in detail.\\[1em]
%
\textbf{Implementation of springs} - The basic concept of a simple spring will be implemented.\\[1em]
%
\textbf{Implementation of objects} - Various soft objects such as a rope, a flag or a jelly will be created to present the softness. Animation and the influence of further forces such as gravity, wind or collisions will enhance the viewing experience.\\[1em]
%
\textbf{Improvement to natural scene} - With additional power provided by the OpenGL engine \cite{opengl} we will add texture, lightning and other renders to create the impression of a realistic scene.\\[1em]
%
\textbf{Code implementation} - We will use the Python programming language, along with the PyOpenGL binding for OpenGL. It provides GL, GLU and GLUT.
%

%\section{Introduction}
%\section{Theory}
%\section{Implementation}
%\section{Experimental Results}
%\section{Conclusion}
%\section{Outlook}

%http://www.ibiblio.org/e-notes/VRML/blaxxun/gallery.htm

%
% Literature / References
%
%\newpage
\nocite{hill}
\nocite{rogersadams}
\nocite{dam}
\nocite{PBDM}
\nocite{baker}
\nocite{bakerGL}
\nocite{IASDO}
\nocite{LSCS}
\nocite{DCMSM}
\nocite{gama}
\nocite{wiki}
\nocite{hair}
\nocite{pyGL}
%\nocite{}
\bibliographystyle{alpha}
\bibliography{gc-bib}

%
% End of File
%
\end{document}
